
% Abstract File

\chapter*{\centering{\uppercase{Abstract}}}


%Facilitating user design of 3D contents remains a challenge in geometric modeling. Instead of intuitively creating models from scratch, a more popular trend is to extract intrinsic patterns from exemplar inputs (e.g., shape collections and sketches), to produce creative models while preserving the induced constraints. The modeling process contains two main stages, analysis and synthesis. The analysis of input models is usually performed on the part level, especially for man-made objects that can be decomposed into several semantic parts, such as the seat and handles of a bicycle. The synthesis stage recombines parts of shapes to generate new models that usually have topological or geometric variations. %
%After that, a validity check is sometimes necessary to ensure the constraints are satisfied with user guided design requirements. Recent work focus on either extracting synthesis rules from a collection of shapes, or deforming models with certain patterns, e.g., repetitive elements, symmetry structures, etc.
%Using a graph representation of shape elements and their relations, we extract and present controllable components to users for supporting designs.

%Various modeling tools have been developed for serving architectural design and artistic creation. However, these tools are usually complicated and professional training are required. In this thesis, we propose several design tools aiming at easing the modeling process from single objects to near regular scenes. We focus on man-made structures and scenes such as buildings and furniture, as these kinds of shapes can be analyzed in component level along with functional relations among parts. New models can be created through recombining shape parts following geometric and topological rules.

Supporting user designs of 3D contents remains a challenge in geometric modeling. Various modeling tools have been developed in recent years to facilitate architectural designs and artistic creations. However, these tools require both modeling skills and raw creativity. Instead of creating models from scratch, one of the most popular choices is to extract intrinsic patterns from exemplar inputs (e.g., shape collections and sketches), to produce creative models while preserving the patterns. The modeling process contains two main stages, analysis and synthesis. The analysis of input models is usually performed at component level, especially for man-made objects that can be decomposed into several semantic parts, such as the seat and handles of a bicycle. The synthesis stage recombines parts of shapes to generate new models that usually have topological or geometric variations.

In this thesis, we propose three design tools aimed at easing the modeling process. We focus on man-made objects and scenes such as buildings and furniture, as the functionality of such shapes can be analyzed at component level. A relation graph, which is commonly used in shape analysis, can then be built to represent the input shapes. In our work, the graph nodes denote the elements of a model (i.e., rooms, shape parts, and strokes respectively), while the edges capture the inner relations between connected elements. With the graph representation, we extract and present controllable components to users for supporting designs.

%can be induced the components and functional relations between the components New models can be created through recombining shape parts following geometric and topological rules. Using a graph representation of shape elements and their relations, we extract and present controllable components to users for supporting designs.
%Relation graph is commonly used for shape analysis in part-based modeling. In our work, the input is represented by a graph whose nodes denote the elements of an object, while the edges naturally capture the inner relations between connected elements. Instead of adding virtual edges or nodes in order to depict specific patterns of a shape like symmetry, we study the intrinsic connectivity of a shape to reduce additional constraints. The relation graph can be mutated automatically or manipulated by user controls, forming new shapes conforming to the abstracted rules.

We introduce our work from three aspects. First, we propose a framework for supporting interior layout design, which allows users to manipulate the produced floor plans, i.e., changing the scales of rooms and their positions as well. When the user modifies the topology of a layout, the corresponding layout graph is updated and the room geometries are optimized under certain constraints, e.g., user specified scales, the adjacency of rooms, and fabrication considerations (i.e., economic construction cost). In the second work, we introduce replaceable substructures as arrangements of shape components that can be interchanged while ensuring boundary consistency. Based on the shape graphs that encode the structures of input models, we propose new automatic operations to discover replaceable substructures across models or within a model. We enforce a pair of subgraphs matching along their boundaries so that switching two subgraphs results in topological variations. Further, we develop an iteractive system that supports a freeform design by interpreting user sketches. 3D contents can be extracted from input strokes with or without user annotations. Our system accepts user strokes, analyzes their contacts and vanishing directions with respect to an anchored image, and projects 2D strokes to 3D space via a multi-stage optimization on spatial canvas selection.










%\doublespacing
%\addcontentsline{toc}{chapter}{Abstract}
%
%\begin{center}
%\Large {Your Thesis Title\\
%Your Full Name}
%\end{center}
%
%This file contains the KAUST Thesis/Dissertation Template as per the latest guidelines \cite{guidelines}.
%
%An abstract of the dissertation or thesis is mandatory and will be used by abstracting and indexing services to provide access to your complete work. The page heading of the abstract is simply the word "ABSTRACT" all in capital letters, centered within the margins at the top of the page. The abstract would be numbered 4 at the top or implicitly at the bottom (if a copyright page is included).
%
%\begin{itemize}
%\item Inclusion of the abstract is mandatory.
%\item	The abstract must provide a succinct and informative summary or synopsis of your work, including: a brief background or introduction; the research area and purpose; the procedures or methods used; the findings or results; and the conclusions.
%\item	Do not exceed 350 words.
%\item	Graphs, mathematical formulas, diagrams, charts, tables or illustrations should not be included.
%\item	Print on one side of the paper only, double-spaced.  Margins must be maintained.
%\item	Avoid abbreviations and acronyms.
%
%\end{itemize}

% Copyright 2010 Imran Shafique Ansari
% Contact Email: imran.ansari@kaust.edu.sa
% Contact Number: +966 59 897 1005
