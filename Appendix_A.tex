
% Appendix A File

\refstepcounter{chapter}%
\chapter*{\thechapter \quad User annotations}
\label{appendixA}

\mypara{Shape segmentation}
Replaceable substructures indicate a group of mesh segments whose boundary matches another group. The boundary is defined by a sequence of connected edges with associated types. We do not assign labels to edges directly, but through the types of the two associated shape parts. Given an input model, we first extract the intrinsic grouping information. We then allow the user label the component information by simply coloring selected elements. For example, the segmentation of a robot model in Figure~\ref{fig:segmentation} is implemented by grouping small parts and then labeling.

\begin{figure}[b!]
  \vnudge
  \includegraphics[width=\columnwidth]{./images/appendix/segmentation.pdf}
  \caption{Given the initial segmentation of an input model (left), the user can select a group of components and assign a color to the group indicating a new shape part (right).}
  \label{fig:segmentation}
\end{figure}

\mypara{Sketch statistics} It is interesting to see that artists usually only use sketch and erase tools as they draw on paper. They try to avoid the ambiguity by changing viewpoints and then sketching. Even without any annotations, our system can still create good results approximating 3D structures, see Figure \ref{fig:teaser}, \ref{fig:design_sketches}. Table~\ref{tbl:statistics} presents the number of strokes, the number of connectors, user annotations, and the average modeling time for the various examples shown in the paper. The majority of the time was spent on sketching using mouse as input. We tried a version using digital pen input, however, the interface is still indirect (the user needs to touch on a separate pad to get the mouse correctly positioned on screen).


\begin{table} \vnudge\centering
\small
\caption{Statistics of our method. %The number of annotations is the number of user override operations used for group and contact editing.
Time includes time for sketching, grouping, annotation and optimization.
} \label{tbl:statistics}
\begin{tabular}{l|c|c|c|c}
\hline
  % after \\: \hline or \cline{col1-col2} \cline{col3-col4} ...
  Figure & \# strokes & \# connec. & \# annotation & time (m) \\
  \hline
\ref{fig:teaser} & 155 & 241 & 0 & 36 \\
\ref{fig:pipeline} & 50 & 34 & 3 &  5 \\
\ref{fig:optimization} &  37 & 12 & 1 & 3\\
\ref{fig:design_sketches} (c) & 71 & 91 & 0 & 10 \\
\ref{fig:sketchupEval} (top) & 79 & 49 & 5 & 10 \\
\ref{fig:results} (top) &  162 & 96  & 5 & 30\\
\hline
\end{tabular}
\end{table}


In the simple examples (i.e., with no or minimal occlusion), the automatic grouping and connector suggestions were sufficient. In other cases, the user had to edit a few grouping or connector suggestions. Such editing and erasing of falsely detected contacts, however, is fairly easy to perform since the false detections occur mostly at occlusion and improper stroke positions (even though the markers might appear to be dense). It typically took users less than 1 minute to investigate and correct such grouping and contacts. Besides, our system allows the user to assign ground touch to the strokes. In experiments, the ground-touch relation is mostly assigned 1 per example.After the user finishes annotation, our selection and optimization algorithm runs at interactive rate. The MRF selection takes longer, about 2-3 seconds per 100 strokes. The quadratic programming is less than a second thanks to the Levenberg-Marquardt algorithm and a good initialization from the selection stage.




%Detailed experimental procedures, data tables, computer programs, etc. may be placed in appendices. This may be particularly appropriate if the dissertation or thesis includes several published papers.

% Copyright 2010 Imran Shafique Ansari
% Contact Email: imran.ansari@kaust.edu.sa
% Contact Number: +966 59 897 1005
